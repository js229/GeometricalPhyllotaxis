

\documentclass[a4paper]{article}
\usepackage{Sweave}
\bibliographystyle{plain}



\usepackage{float}
\newcommand{\uvec}{{\bf u}}
\newcommand{\mvec}{{\bf m}}
\newcommand{\nvec}{{\bf n}}
\newcommand{\vvec}{{\bf v}}
\newcommand{\half}{\textstyle{\frac{1}{2}}}

\begin{document}

\title{The inverse lattice}
\author{Jonathan Swinton}
\maketitle

%\section{Housekeeping}
\begin{Schunk}
\begin{Sinput}
> library(GeometricalPhyllotaxis)
> library(grid)
> library(xtable)
> library(showtext)
> showtext.auto()
> 
\end{Sinput}
\end{Schunk}
\begin{Schunk}
\begin{Sinput}
> matrix.test <- matrix(c(2,1,-1,2),ncol=2,byrow=TRUE)
> PM.test <- as(matrix.test,"PhyllotaxisMatrix")
> 
> 
\end{Sinput}
\end{Schunk}
\section{The inverse lattice}

The inverse lattice is the lattice whose matrix is the transpose of the inverse
of the lattice matrix:

\begin{Schunk}
\begin{Sinput}
> inverseLatticeMatrix <- function(from) {
+ 	imat <- from@Coord
+ 	nmat <- imat*from@nScale # hoping this will make integers...
+ 	if (sum(abs(nmat-floor(nmat)))>1e-12) {
+ 		stop("Numerical rounding errors")
+ 	}
+ 	invtmat <- matrix(c( nmat[2,2], -nmat[1,2], -nmat[2,1],nmat[1,1]),ncol=2,byrow=TRUE) # inverse and transpose
+ 	delta <- (nmat[1,1]*nmat[2,2] -nmat[2,1]*nmat[1,2])
+ 	if (delta<0) { delta <- -delta; invtmat <- -invtmat }
+ 	object <- new("PhyllotaxisMatrix", nScale=from@dScale,dScale=delta, Coord=invtmat )
+ 	object <- GeometricalPhyllotaxis:::extractHCFIntoRise(object)
+ 	object@Rho <- GeometricalPhyllotaxis:::getRho.PM(object,Jugacy=1)
+ 	object
+ }
\end{Sinput}
\end{Schunk}
This is not $2\pi\rho$ periodic.


Some plot methods are defined.


 <<xpIPM,eval=TRUE>>=
 iPM.test <-inverseLatticeMatrix (PM.test)
 print(PM.test)
 print(iPM.test)
 #plotInverse(iPM.test,y=4,doNewPage=TRUE)
 stopifnot(all.equal(as(inverseLatticeMatrix (iPM.test),"matrix"),as(PM.test,"matrix")))
 GeometricalPhyllotaxis:::makePrincipalPhyllotaxisMatrix (iPM.test)
 @


\begin{Schunk}
\begin{Sinput}
> plotLatticeAndInverse(PM.test,iPM.test,10,3)
\end{Sinput}
\end{Schunk}
\clearpage
\section{K4}
\begin{Schunk}
\begin{Sinput}
> iPK4 <- GeometricalPhyllotaxis:::makePrincipalPhyllotaxisMatrix (as(matrix(c(10,5,5,-7),nrow=2,byrow=TRUE),"PhyllotaxisMatrix"))
> iiPK4 <- inverseLatticeMatrix (iPK4)
> PK4 <- GeometricalPhyllotaxis:::makePrincipalPhyllotaxisMatrix (iiPK4)
> print(PK4)
\end{Sinput}
\begin{Soutput}
An object of class "PhyllotaxisMatrix"
Slot "nScale":
[1] 1

Slot "dScale":
[1] 95

Slot "Coord":
     [,1] [,2]
[1,]    5    7
[2,]   10   -5

Slot "Rho":
[1] 0.1591549

Slot "bottomOrigin":
[1] TRUE

Slot "lhsOrigin":
[1] TRUE

Slot "L":
[1] 50
\end{Soutput}
\end{Schunk}
\begin{figure}\begin{center}
\begin{Schunk}
\begin{Sinput}
> plotLatticeAndInverse(PK4,iPK4,y=1,35,wire=4)
\end{Sinput}
\end{Schunk}
\caption{Inverse matrix with the same lattice as AMT-K4}
\end{center}
\end{figure}

\clearpage
\section{K5}
\begin{Schunk}
\begin{Sinput}
> iPK5 <- (as(matrix(c(12,-2,8,6),nrow=2,byrow=TRUE),"PhyllotaxisMatrix"))
> print(iPK5)
\end{Sinput}
\begin{Soutput}
An object of class "PhyllotaxisMatrix"
Slot "nScale":
[1] 2

Slot "dScale":
[1] 1

Slot "Coord":
     [,1] [,2]
[1,]    6   -1
[2,]    4    3

Slot "Rho":
[1] 14.00563

Slot "bottomOrigin":
[1] TRUE

Slot "lhsOrigin":
[1] TRUE

Slot "L":
[1] 50
\end{Soutput}
\begin{Sinput}
> K5 <- (inverseLatticeMatrix (iPK5))
> PK5 <- GeometricalPhyllotaxis:::makePrincipalPhyllotaxisMatrix (K5)
> print(PK5)
\end{Sinput}
\begin{Soutput}
An object of class "PhyllotaxisMatrix"
Slot "nScale":
[1] 2

Slot "dScale":
[1] 88

Slot "Coord":
     [,1] [,2]
[1,]    3    1
[2,]   -1    7

Slot "Rho":
[1] 0.07957747

Slot "bottomOrigin":
[1] TRUE

Slot "lhsOrigin":
[1] TRUE

Slot "L":
[1] 50
\end{Soutput}
\end{Schunk}
\begin{figure}\begin{center}
\begin{Schunk}
\begin{Sinput}
> plotLatticeAndInverse(PK5,iPK5,y=1,35,wire=4)
\end{Sinput}
\end{Schunk}
\caption{Inverse matrix with the same lattice as AMT-K5}
\end{center}
\end{figure}

\clearpage
\section{K6}
\begin{Schunk}
\begin{Sinput}
> iK6a <- (as(matrix(c(40,36,80,-30),nrow=2,byrow=TRUE),"PhyllotaxisMatrix"))
> iPK6a <- GeometricalPhyllotaxis:::makePrincipalPhyllotaxisMatrix (iK6a,trace=TRUE)
\end{Sinput}
\begin{Soutput}
=====
|a	b|  = 6  /1   *  |6.66667	  6|
|c	d|               |13.3333	 -5|
Not a principal vector matrix
Opposed
u.v: 2120; u^2 2896; v^2: 7300; (u-v)^2: 5956; (u+v)^2: 14436
Delta^2 1.66464e+07 
(m,n)=(6,-5); (1*6)-(-1*-5) = 1
Rise 6
Minimal cylinder radius 108.225 circumference 680 
Divergence 5.91359 (338.824 degrees; Jean d 0.441176)
Visible opposed divergence interval (1.0472,1.25664)
=====
R
\end{Soutput}
\begin{Sinput}
> print(iPK6a)
\end{Sinput}
\begin{Soutput}
An object of class "PhyllotaxisMatrix"
Slot "nScale":
[1] 6

Slot "dScale":
[1] 1

Slot "Coord":
          [,1] [,2]
[1,]  6.666667    6
[2,] -6.666667   11

Slot "Rho":
[1] 649.3522

Slot "bottomOrigin":
[1] TRUE

Slot "lhsOrigin":
[1] TRUE

Slot "L":
[1] 50
\end{Soutput}
\begin{Sinput}
> K6a <- (inverseLatticeMatrix (iPK6a))
> PK6a <- GeometricalPhyllotaxis:::makePrincipalPhyllotaxisMatrix (K6a)
> print(PK6a)
\end{Sinput}
\begin{Soutput}
An object of class "PhyllotaxisMatrix"
Slot "nScale":
[1] 4

Slot "dScale":
[1] 4080

Slot "Coord":
     [,1] [,2]
[1,] 10.0   10
[2,] 16.5   -9

Slot "Rho":
[1] 0.03978874

Slot "bottomOrigin":
[1] TRUE

Slot "lhsOrigin":
[1] TRUE

Slot "L":
[1] 50
\end{Soutput}
\begin{Sinput}
> PK6a.2 <- ((as(matrix(c(47,32,-25,32),nrow=2,byrow=TRUE),"PhyllotaxisMatrix")))
> print(PK6a.2)
\end{Sinput}
\begin{Soutput}
An object of class "PhyllotaxisMatrix"
Slot "nScale":
[1] 32

Slot "dScale":
[1] 1

Slot "Coord":
         [,1] [,2]
[1,]  1.46875    1
[2,] -0.78125    1

Slot "Rho":
[1] 366.693

Slot "bottomOrigin":
[1] TRUE

Slot "lhsOrigin":
[1] TRUE

Slot "L":
[1] 50
\end{Soutput}
\begin{Sinput}
> iPK6a.2 <-  GeometricalPhyllotaxis:::makePrincipalPhyllotaxisMatrix (inverseLatticeMatrix(PK6a.2))
> 
\end{Sinput}
\end{Schunk}
\begin{figure}\begin{center}
\begin{Schunk}
\begin{Sinput}
> plotLatticeAndInverse(PK6a,iPK6a,y=.3,35,wire=4)
\end{Sinput}
\end{Schunk}
\caption{Inverse matrix with the same lattice as AMT-K6}
\end{center}
\end{figure}

The lattice parameters:
\begin{center}
\begin{tabular}{llllllll}
Figure 	& Wirewidth &  Cylinder 			& Ring  & Scaled  &Expected  	& Parastichy  & Inverse
\\
 				&  					&  cf &  height &  cf & cf 	&  numbers &  Lattice
\\
 				& 	 				&  (grid squares) &  	 &  	 & 	&   &  matrix
\\
K4  & 5   &30 & 24 & 20 & 20 & (1,2) & $ \left( \begin{array}{ll}	7 & 5 \\ -5  & 10 \end{array} \right) $
\\
K5 & 4  & 24 & 16 & 24 & 25 & (2,1) & $ \left( \begin{array}{ll}	4 & -8 \\ 8  & 6  \end{array}  \right)$
\\
K6 & 10/3  &  28 & 16 & 28 & 30 & (1,2) & $ \frac{1}{6}\left( \begin{array}{ll}	40 & 36 \\ 80  & -30  \end{array} \right)$
%\
% 				&   				&  								&  &  &  		& (,) & $ \frac{1}{6}\left( \begin{array}{ll}	20 & -36 \\ 20  & 69  \end{array} $
\end{tabular}
\end{center}

\section{Illustrations}

\begin{Schunk}
\begin{Sinput}
> tau = (sqrt(5)+1)/2
> Fibonacci.Divergence <- 2*pi/tau^2
> PF.fig <- newPhyllotaxisGenetic(Rise=2,
+ 	Divergence=Fibonacci.Divergence,Rho=5 , Jugacy=1)
> PG.fig <- GeometricalPhyllotaxis:::makePrincipalPhyllotaxisMatrix (as(PF.fig,"PhyllotaxisMatrix"))
\end{Sinput}
\end{Schunk}
\begin{Schunk}
\begin{Sinput}
> plotPhyllotaxis(PG.fig,y=50,doAxes=2,doNumbers=TRUE,plotPrincipals=10)
> plot.RiseLabel.grid(PG.fig,L=50)
> #savePlot("ams-fig-lattice-parameters",type="jpg")
\end{Sinput}
\end{Schunk}

\bibliography{GPpackage}
\end{document}

